\documentclass{beamer}

\usepackage{minted}

\begin{document}
\begin{frame}[fragile]
\frametitle{from clojuredocs with a bit of destructuring}
\begin{minted}[
frame=lines,
framesep=2mm,
baselinestretch=1.2,
%bgcolor=LightGray,
fontsize=\small,
linenos
]{clojure}
(defn lazy-trialdivision
  ([] (lazy-trialdivision (iterate inc 2)))
  ([[prime & nums]]
    (cons prime
          (lazy-seq
              (lazy-trialdivision (filter
                                     #(not= 0 (mod % prime))
                                     nums))))))
\end{minted}

\end{frame}

\begin{frame}[fragile]
\frametitle{java interop -- imperative style}
\begin{minted}[
frame=lines,
framesep=2mm,
baselinestretch=1.2,
%bgcolor=LightGray,
fontsize=\small,
linenos
]{clojure}
(defn java-sieve [n]
  "returns a BitSet with bits set for each prime up to n"
  (let [bs (java.util.BitSet. n)]
    (.flip bs 2 n)
    (doseq [p (range 2 (Math/sqrt n))]
      (if (.get bs p)
        (doseq [q (range (* p p) n p)] (.clear bs q))))
    bs))
\end{minted}

\end{frame}

\begin{frame}[fragile]
\frametitle{wheel factorization}
\begin{minted}[
frame=lines,
framesep=2mm,
baselinestretch=1.2,
%bgcolor=LightGray,
fontsize=\small,
linenos
]{clojure}
(def wheel-sieve
  (concat
   [2 3 5 7]
   (lazy-seq
    (let [primes-from
          (fn primes-from [n [f & r]]
            (if (some #(zero? (rem n %))
                      (take-while #(<= (* % %) n) wheel-sieve))
              (recur (+ n f) r)
              (lazy-seq (cons n (primes-from (+ n f) r)))))
          wheel (cycle [2 4 2 4 6 2 6 4 2 4 6 6 2 6  4  2
                        6 4 6 8 4 2 4 2 4 8 6 4 6 2  4  6
                        2 6 6 4 2 4 6 2 6 4 2 4 2 10 2 10])]
      (primes-from 11 wheel)))))
\end{minted}
\end{frame}


\begin{frame}[fragile]
\frametitle{Genuinely lazy eratosthenes sieve}
\begin{minted}[
frame=lines,
framesep=2mm,
baselinestretch=1.2,
%bgcolor=LightGray,
fontsize=\small,
linenos
]{clojure}
(defn lazy-eratosthenes "genuinely lazy Eratosthenes sieve"
  []
  (let [reinsert (fn [table composite factor]
                   (update-in table
                              [(+ factor composite)]
                              conj factor))]
    (defn primes-step [table d]
      (if-let [factors (get table d)]
        (recur (reduce #(reinsert %1 d %2)
                       (dissoc table d) factors)
               (inc d))
        (lazy-seq (cons d (primes-step (assoc table
                                         (* d d)
                                         (list d))
                                       (inc d))))))
    (primes-step {} 2)))
\end{minted}
\end{frame}

\end{document}
